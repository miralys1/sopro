\begin{tcolorbox}
Je nach Projekt und Produkt sind bestimmte Eigenschaften der zu entwickelnden Software wichtiger als andere.
Diese einzelnen Kriterien werden üblicherweise mittels einer Tabelle, wie z.B. in~\autoref{tabelle:qualitaetsanforderungen} zu sehen, notiert.
Sie dienen u.a. dazu dem gegenüber Kunden einen hohen Aufwand für ein bestimmtes Kriterium deutlich zu machen.
\end{tcolorbox}

\begin{table}[h]
	\centering
	\begin{tabularx}{\textwidth}{l c c c c}
		\rowcolor[HTML]{C0C0C0} 
		& \textbf{sehr wichtig} & \textbf{wichtig} & \textbf{weniger wichtig} & \textbf{unwichtig} \\
		Robustheit &  &  &  & x \\
		\rowcolor[HTML]{E7E7E7} 
		Zuverlässigkeit &  &  &  & x \\
		Wartbarkeit &  &  &  & x \\
		\rowcolor[HTML]{E7E7E7} 
		Erweiterbarkeit &  &  &  & x \\
		Benutzerfreundlichkeit &  &  &  & x \\
		\rowcolor[HTML]{E7E7E7} 
		Effizienz &  &  &  & x \\
		Anpassbarkeit &  &  &  & x \\
		\rowcolor[HTML]{E7E7E7} 
		Kompatibilität &  &  &  & x \\
		Sicherheit &  &  &  & x
	\end{tabularx}
	\caption{Qualitätsanforderungen}
	\label{tabelle:qualitaetsanforderungen}
\end{table}