In folgendem Kapitel werden verschiedene Benutzungsszenarien der Software beschrieben und das erwartete Verhalten definiert.
In Tabelle \ref{fig:testfaelle-androidapp} wird das Verhalten der Android-App definiert, in Tabelle \ref{fig:testfaelle-webapp-a} und Tabelle \ref{fig:testfaelle-webapp-b} das der Web-App.


\begin{figure}[!h]
	\newcounter{test}
	\begin{center}
		\begin{tabularx}{\textwidth}{ p{.05\textwidth} | p{.2\textwidth} | X | X }
			\textbf{Nr.} & \textbf{Anwendungsfall ID} & \textbf{Szenario} & \textbf{Erwartetes Verhalten} 
			\\ \hline
			\stepcounter{test}\arabic{test}& App-1 & 
			Beim Einloggen wird ein nicht registrierter Nutzername eingegeben.& Es wird eine passende Fehlermeldung angezeigt, ein Login kann erneut versucht werden.
			\\ \hline
			\stepcounter{test}\arabic{test} & App-1 & 
			Das zu einem registrierten Nutzernamen eingegebene Passwort ist nicht korrekt.& Es wird eine passende Fehlermeldung angezeigt, ein Login kann erneut versucht werden.
			\\ \hline
			\stepcounter{test}\arabic{test} & App-1 & 
			Während des Loginvorgangs ist der Server nicht mehr erreichbar. & Eine passende Fehlermeldung wird angezeigt.\\ \hline
			\stepcounter{test}\arabic{test} & App-1 & 
			Ein registrierter Nutzername und das dazu passende Passwort werden eingegeben.& Der Login ist erfolgreich und es wird die erweiterte Listenansicht der Kompositionen angezeigt.
			\\ \hline
			\stepcounter{test}\arabic{test} & App-2 & 
			Die Listenansicht wird von einer eingeloggten Person angefordert. & Eine Liste der für diese Person einsehbaren Kompositionen wird angezeigt.
			\\ \hline
			\stepcounter{test}\arabic{test} & App-2 & 
			Die Listenansicht wird von einer nicht eingeloggten Person angefordert. & Es erscheint eine Listenansicht der öffentlichen Kompositionen.
			\\ \hline
			\stepcounter{test}\arabic{test} & App-3 & 
			Es wird die Detailansicht einer Komposition angefordert. & Die Komposition wird mit den Kompatibilitätshinweisen zwischen den einzelnen Diensten graphisch angezeigt.
			\\ \hline
			\stepcounter{test}\arabic{test} & App-4 & 
			Das Versenden einer Komposition als PDF schlägt fehl. & Es wird eine passende Fehlermeldung angezeigt.
			\\ \hline
			\stepcounter{test}\arabic{test} & App-4 & 
			Das Versenden einer Komposition als PDF war erfolgreich. & Es wird eine Bestätigung angezeigt. 
			\\ \hline
		\end{tabularx}	
	\end{center}
	
	
	
	
	\caption{Beschreibung der Testfälle für die Android-App}
	\label{fig:testfaelle-androidapp}
\end{figure}


\begin{figure}[!h]
	\begin{center}
		\begin{tabularx}{\textwidth}{ p{.05\textwidth} | p{.2\textwidth} | X | X }
			\textbf{Nr.} & \textbf{Anwendungsfall ID} & \textbf{Szenario} & \textbf{Erwartetes Verhalten} \\ \hline
			1 & Web-1/2 & Ein Nutzender ohne Administrator-Rechte versucht, einen Dienst einzufügen. & Die Aktion wird verweigert und es wird eine passende Fehlermeldung angezeigt.\\ \hline
			2 & Web-1 & Ein Nutzender mit Administrator-Rechten wählt aus, einen Dienst manuell einzufügen. & Es wird das Eingabeformular für Dienste angezeigt. \\ \hline
			3 & Web-1 & Das Eingabeformular wird syntaktisch korrekt abgeschickt. & Der neue Dienst wird in die Datenbank einfügt. \\ \hline
			4 & Web-1 & Beim Eingabeformular wird das Schlagwortfeld ausgewählt. & Es werden zur Verfügung stehende Schlagwörter angezeigt. \\ \hline
			5 & Web-1 & Das Eingabeformular wird nicht syntaktisch korrekt abgeschickt. & Es wird ein Hinweis angezeigt.\\ \hline
			6 & Web-2 & Eine Datei im passenden JSON-Format wird eingelesen. & Die Dienste der Datei werden in die Datenbank aufgenommen und es wird eine Erfolgsmeldung angezeigt.\\ \hline
			7 & Web-2 & Eine Datei in syntaktisch nicht passenden JSON-Format wird eingelesen. & Es wird eine entsprechende Fehlermeldung angezeigt.\\ \hline
			8 & Web-3 & Ein Nutzender ohne Administrator-Rechte versucht einen Dienst zu bearbeiten. & Es wird eine passende Fehlermeldung angezeigt.\\ \hline
			9 & Web-3 & Ein Dienst wird bearbeitet und dadurch invalide. & Die Änderung wird verweigert.\\ \hline
			10 & Web-3 & Ein Dienst wird korrekt bearbeitet. & Die Änderung wird in die Datenbank übernommen und es erscheint eine Erfolgsnachricht.\\ \hline
			11 & Web-4 & Eine unregistrierte Person möchte eine Komposition erstellen. & Die Aktion wird mit Fehlermeldung verweigert.\\ \hline
			12 & Web-4 & Eine registrierte Person möchte eine Komposition erstellen. & Es wird die Weboberfläche zum Bearbeiten einer Komposition angezeigt.\\ \hline
			13 & Web-4 & Eine erstellte Komposition soll gespeichert werden. & Die Komposition wird mit den angegebenen Einstellungen in die Datenbank übernommen.\\ \hline
	\end{tabularx}

	\end{center}

	\caption{Beschreibung der Testfälle für die Web-App}
	\label{fig:testfaelle-webapp-a}
\end{figure}
\newpage
\begin{figure}[!h]
	\begin{center}
	\begin{tabularx}{\textwidth}{ p{.05\textwidth} | p{.2\textwidth} | X | X }
		\textbf{Nr.} & \textbf{Anwendungsfall ID} & \textbf{Szenario} & \textbf{Erwartetes Verhalten} \\ \hline


			14 & Web-5 & Eine Person mit nicht ausreichenden Rechten versucht, eine Komposition zu bearbeiten. & Die Aktion wird verweigert und es wird eine passende Fehlermeldung ausgegeben. \\ \hline
			15 & Web-5 & Eine Person mit ausreichenden Rechten möchte eine Komposition bearbeiten. & Die Weboberfläche zum Bearbeiten einer Komposition wird angezeigt. \\ \hline
			16 & Web-5 & Eine bearbeitete Komposition soll gespeichert werden. & Die Änderungen werden in die Datenbank übernommen. \\ \hline
			17 & Web-6 & Eine Person möchte eine für sie nicht einsehbare Komposition einsehen. & Es wird eine entsprechende Fehlermeldung angezeigt und die Aktion verweigert. \\ \hline
			18 & Web-6 & Eine Person möchte eine für sie einsehbare Komposition einsehen. & Die gewählte Komposition wird visuell dargestellt. \\ \hline
			
		\end{tabularx}	

	\end{center}
	

	\caption{Beschreibung der Testfälle für die Web-App}
	\label{fig:testfaelle-webapp-b}
\end{figure}
