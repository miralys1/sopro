
\textbf{Funktionale Zielbestimmungen}\newline

\textit{Musskriterium}

\begin{itemize}[leftmargin=4pc]
	\item Die Anwendung muss über eine naive Android-App erreichbar sein.
	\item Die Anwendung muss über eine Website erreichbar sein.
	\item Auf der Website muss die Möglichkeit bestehen Konfigurationen zu erstellen.
	\item Es muss eine REST-Schnittstelle geben, um Dienste einzulesen.
	\item Die Anwendung muss eine Kompatibilitätsprüfung durchführen.
	\item Die Speicherung von erstellten Konfigurationen muss möglich sein.
	\item Die Android-App muss erstellte Konfigurationen anzeigen können.
	\item Benutzer der Anwendung müssen sich registrieren können.
	\item Benutzer der Anwendung müssen sich anmelden können.
	\item Beim manuellen Erstellen von Diensten muss das Format überprüft werden.
\end{itemize}

\textit{Sollkriterium}

\begin{itemize}[leftmargin=4pc]
	\item Website soll die Möglichkeit bieten Dienste manuell zu erstellen. (für Admins?)
	\item Benutzer sollen verschiedene Rechte haben.
	\item Konfigurationen sollen mit anderen Benutzern geteilt werden können.
	\item Veränderbarkeit von Konfigurationen durch andere Benutzer soll eingestellt werden können.
	\item In der Android-App soll die Möglichkeit bestehen, Konfigurationen zu verschicken.
\end{itemize}

\textit{Kannkriterium}

\begin{itemize}[leftmargin=4pc]
	\item Android-App kann einzelne Dienste anzeigen.
	\item Es können Alternativen angezeigt werden, wenn Dienste nicht kompatibel sind.
	\item Es gibt eine Suchfunktion für Dienste und Konfigurationen
	\item Auf der Website können Konfigurationen verschickt werden
\end{itemize}

\textit{Abgrenzungskriterium}

\begin{itemize}[leftmargin=4pc]
	\item In der Android-App können keine Konfigurationen erstellt werden.
	\item In der Android-App können keine Dienste erstellt werden.
	\item In der Android-App können keine Benutzer registriert werden.
	\item Beim manuellen Erstellen von Diensten wird nicht geprüft, ob der Inhalt korrekt ist.
	\item Es gibt keine benutzerspezifische Auswahl von Diensten.\\
\end{itemize}


\textbf{nicht funktionale Zielbestimmungen}\newline

\textit{Musskriterium}

\begin{itemize}[leftmargin=4pc]
	\item Die Website muss mit Java-Spring erstellt werden.
	\item Die Android-App muss ab der Android Version 6 benutzbar sein.
	\item In der Android-App muss die URL änderbar sein.
	\item Es muss eine grafische Rückmeldung über die Kompatibilität vn Diensten geben.
\end{itemize}

\textit{Sollkriterium}

\begin{itemize}[leftmargin=4pc]
	\item Die Kommunikation zwischen Website und Android-App soll veschlüsselt sein.
	\item Es soll eine Drag and Drop Benutzeroberfläche geben.
	\item Die Benutzeroberfläche soll einfach und intuitiv sein.
	\item Es soll eine geeignete Lizenz für die Anwendung geben.
\end{itemize}