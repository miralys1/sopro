Zu den Benutzern werden folgende Daten gespeichert:
\begin{itemize}
	\item Name
	\item ID
	\item Passwort
	\item Erstellte Kompositionen
	\item Kompositionen, für die er Leserechte besitzt
	\item Kompositionen, für die er Schreibrechte besitzt
	\item Benutzergruppen, zu welchen er gehört.
\end{itemize}
Zu den Diensten werden folgende Daten gespeichert:
\begin{itemize}
	\item Name
	\item Organisation
	\item Version
	\item ID
	\item Schlagworte
	\item gültige Eingabeformate
	\item gültige Ausgabeformate
\end{itemize}
Zu den Kompositionen werden folgende Daten gespeichert:
\begin{itemize}
	\item Autor
	\item Enthaltene Dienste
	\item Benutzergruppen mit Leserechten
	\item Benutzergruppen mit Schreibrechten
\end{itemize}

Zu den Ein- und Ausgabeformaten werden folgende Daten:
\begin{itemize}
	\item Typ des Eingabeformats (z.B JSON)
	\item Version
	\item Abwärtskompatibilität\\
\end{itemize}
Bei der Abwärtskompatibilität wird zwischen \textit{strict} und \textit{flexible} unterschieden. Mit \textit{flexible} markierte Formate sind abwärtskompatibel, mit \textit{strict} gekennzeichnete hingegen nicht.
