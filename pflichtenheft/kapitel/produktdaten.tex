Um die Funktionsweise des Dienstes zu realisieren, ist es notwendig die hier genannten Daten zu speichern.

Zu den BenutzerInnen werden folgende Daten gespeichert:
\begin{itemize}
	\item Anrede
	\item Name
	\item E-Mail-Adresse
	\item Passwort
	\item Erstellte Kompositionen
	\item Kompositionen, die eingesehen oder geändert werden dürfen
	\item vorhandene Administratorrechte
\end{itemize}
Zu den Diensten werden folgende Daten gespeichert:
\begin{itemize}
	\item Name
	\item Organisation
	\item Version
	\item Schlagworte
	\item gültige Eingabeformate (ggf. auch keines)
	\item gültige Ausgabeformate (ggf. auch keines)
\end{itemize}
Zu den Kompositionen werden folgende Daten gespeichert:
\begin{itemize}
	\item Autor
	\item Enthaltene Dienste
	\item Kompatibilitätsbeziehungen der Dienste
	\item Sichtbarkeit (öffentlich, nur bestimmte Benutzer)
	\item Zum Ändern berechtigte BenutzerInnen
\end{itemize}

Zu den Ein- und Ausgabeformaten werden folgende Daten gespeichert:
\begin{itemize}
	\item Name des Eingabe- und Ausgabeformats (z.B JSON)
	\item Version
	\item Abwärtskompatibilität\\
\end{itemize}
Bei der Abwärtskompatibilität wird zwischen \textit{strict} und \textit{flexible} unterschieden. Mit \textit{flexible} markierte Formate sind abwärtskompatibel, mit \textit{strict} gekennzeichnete hingegen nicht.
