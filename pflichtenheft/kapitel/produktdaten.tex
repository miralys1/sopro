Folgende Benutzerdaten werden gespeichert:
\begin{itemize}
	\item Name
	\item ID
	\item Passwort
	\item Erstellte Kompositionen
	\item Kompositionen, für die er Leserechte besitzt
	\item Kompositionen, für die er Schreibrechte besitzt
	\item Benutzergruppe, zu welcher er gehört.
\end{itemize}

Zu den Diensten werden diese Daten gespeichert:
\begin{itemize}
	\item Identifikationsnummer
	\item Name
	\item Organisation
	\item Version
	\item Eingabeformat
	\item Schlagworte
\end{itemize}

Zu den von Nutzern erstellten Dienstkompositionen werden
\begin{itemize}
	\item Autor
	\item Enthaltene Dienste
	\item Benutzergruppen mit Leserechten
	\item Benutzergruppen mit Schreibrechten
\end{itemize}
gespeichert.

Außerdem werden Informationen zu den verwendeten Ein- und Ausgabeformaten gespeichert. Dabei handelt es sich um folgende:
\begin{itemize}
	\item Typ des Eingabeformats (z.B JSON)
	\item Version
	\item Abwärtskompatibilität
\end{itemize}

Bei der Abwärtskompatibilität wird zwischen "strict" und "flexible" unterschieden. Mit "flexible" markierte Formate sind abwärtskompatibel, mit "strict" gekennzeichnete hingegen nicht.
