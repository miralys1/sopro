\begin{tcolorbox}
In diesem Kapitel werden die folgenden Punkte erläutert. Eine jeweilige Unterteilung in Client und Server ist sinnvoll.
\begin{enumerate}
	\item \textit{Software:} Welche Software (Betriebssystem, Datenbanken, Webserver, externe Programme, etc.) ist auf den Zielsystemen für einen Betriebseinsatz erforderlich?
	\item \textit{Hardware:} Welche Hardware ist für den Produkteinsatz notwendig? Insbesondere Mindestanforderungen sind hier zu erwähnen.
	\item \textit{Orgware:} Umfasst organisatorische Anforderungen an die Produktumgebung, welche nicht unter die ersten beiden Kategorien fallen. 
	Dieser Punkt ist stark abhängig vom Projekt und kann auch nur weniger interessante Informationen, wie z.B. Zugang zum Internet umfassen.
	\item \textit{Produktschnittstellen:} Welche Schnittstellen werden zur Laufzeit von dem zu entwickelnden System genutzt (kurze textuelle Beschreibung)?
\end{enumerate}

\noindent Die einzelnen Abschnitte der Produktumgebung können als Fließtexte oder Absätze / Paragraphen mit ganzen Sätzen geschrieben werden.
\end{tcolorbox}