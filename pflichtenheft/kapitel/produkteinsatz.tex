\begin{tcolorbox}
In diesem Kapitel werden die folgenden drei Punkte erläutert:
\begin{enumerate}
	\item \textit{Anwendungsgebiete:} Was ist der Zweck des Produkts?
	\item \textit{Zielgruppen:} Für welche Benutzer (oder auch Rollen) ist das Produkt bestimmt?
	Welche Qualifikationen brauchen die Personen?
	\item \textit{Betriebsbedingungen:} Ist eine bestimmte physikalische Umgebung notwendig? 
	Wie ist die tägliche Betriebszeit des Produkts? 
	Automatische oder manuelle Datensicherung? 
	Autonomer oder beobachtender Betrieb?
\end{enumerate}

\noindent Der Teile des Produkteinsatzes werden üblicherweise als Fließtexte geschrieben.
\end{tcolorbox}
\section*{Anwendungsgebiete}
Der BIMSWARM Composer dient Anwender*Innen des BIMSWARM-Ökosystems dazu, Kombinationen von Diensten auf ihre Kompatibilität zu prüfen. Über das Ergebnis dieser Prüfung gibt das System grafisch eine Rückmeldung. Gleichzeitig lässt sich mit ihm ein Überblick über die vorhandenen atomaren Dienste erlangen sowie Kombinationen verwalten. 

\section*{Zielgruppen}

Primär richtet sich der BIMSWARM Composer an Benutzer*Innen aus der Architektur- und Baubranche, ohne dass weiteres Zusatzwissen über die interne Funktionsweise des Systems vorausgesetzt ist. Allerdings ist Wissen über übliche Services, die im BIMSWARM-Ökosystem genutzt werden, hiflreich, um Kombinationen zu erstellen.

Es sind zwei Rollen vorgesehen: Zum einen an Standard-User, die sowohl Dienste zu Kombinationen zusammensetzen als auch diese prüfen, speichern und teilen, zum anderen an Admin-User, die zusätzlich Dienste anlegen können. Ein Admin muss hierfür das von Adesso spezifizierte Format für Dienste kennen.

\section*{Betriebsumgebungen}

Zum Betrieb des Systems ist ein Server-Computer, auf dem die Webanwendung laufen kann, zwingend notwendig. Wenn die Nutzung der App gewünscht ist, ist hierfür ein Android-Mobilgerät in Form eines Tablets oder Smartphones notwendig. Für die auf einem Server (beispielsweise in Form von Apache Tomcat) laufende Anwendung ist ganztägige Uptime und autonomes Laufen gewährleistet. Es findet automatische Datensicherung statt.




