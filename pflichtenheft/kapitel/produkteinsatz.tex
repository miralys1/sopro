\section*{Anwendungsgebiete}
Der SWARM Composer dient AnwenderInnen des SWARM-Ökosystems dazu, Kompositionen von Diensten auf ihre Kompatibilität zu prüfen. Über das Ergebnis dieser Prüfung gibt das System grafisch eine Rückmeldung. Gleichzeitig lässt sich mit ihm ein Überblick über die vorhandenen atomaren Dienste erlangen sowie Kompositionen verwalten. 

\section*{Zielgruppen}

Primär richtet sich der SWARM Composer an BenutzerInnen aus der Architektur- und Baubranche, ohne dass weiteres Zusatzwissen über die interne Funktionsweise des Systems vorausgesetzt ist. Allerdings ist Wissen über übliche Services, die im SWARM-Ökosystem genutzt werden, hiflreich, um Kombinationen zu erstellen.

Es sind zwei Rollen vorgesehen: Zum einen Standard-NutzerInnen, die sowohl Dienste zu Kompositionen zusammensetzen als auch diese prüfen, speichern und teilen, zum anderen AdministratorInnen, die zusätzlich Dienste anlegen können. Hierfür muss das von Adesso spezifizierte Format für Dienste bekannt sein.

\section*{Betriebsumgebungen}

Zum Betrieb des Systems ist ein Server-Computer, auf dem die Webanwendung laufen kann, zwingend notwendig. Wenn die Nutzung der App gewünscht ist, ist hierfür ein Android-Mobilgerät in Form eines Tablets oder Smartphones notwendig. Für die auf einem Server (beispielsweise in Form von Apache Tomcat) laufende Anwendung ist ganztägige Uptime und autonomes Laufen gewährleistet. Es findet eine automatische Datensicherung statt.




