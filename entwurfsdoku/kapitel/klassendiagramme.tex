\makeatletter
\newcommand*{\compress}{\@minipagetrue}
\makeatother

\begin{figure}[h]
	\centering
	\missingfigure{Klassendiagramm}
	\caption{Klassendiagramm - A}
	\label{fig:klassendiagramm-a}
\end{figure}

\begin{table}[h]
	\centering
	\begin{tabularx}{\textwidth}{p{.2\textwidth} | X} 
		\rowcolor[HTML]{C0C0C0}
		\textbf{Klassenname} & \textbf{Aufgabe} \\
		User & \compress \begin{itemize}
			\item Speicherung der Nutzerdaten, sowie einer künstlichen Datenbank-ID
			\item Eine Flag, die den Adminstatus festlegt
			\item Listen von erstellten, einsehbaren und veränderbaren Kompositionen
		\end{itemize}\\
		\rowcolor[HTML]{E7E7E7}
		Composition & \compress \begin{itemize}
		  \item Speicherung einer künstlichen ID
			\item Speicherung der Kompositionen als Graphen, durch Speicherung der Knoten und Kanten
			\item Speicherung des Urhebers und der Nutzenden mit Zugriffs- bzw. Bearbeitungsrechten
			\item Konvertermethode, zum Erstellen eines detaillierten, versendbaren Objektes
			\item Konvertermethode, zum Erstellen eines reduzierten Objektes
		\end{itemize} \\
		CompositionNode & \compress \begin{itemize}
			\item Speicherung eines Dienstes, für dessen Verwendung im Kompositionsgraphen
			\item Konvertermethode, zum Erstellen eines detaillierten, versendbaren Objektes
		\end{itemize} \\
		\rowcolor[HTML]{E7E7E7}
		CompositionEdge & \compress \begin{itemize}
			\item Speicherung einer gerichteten Kompositionskante als Paar von Kompositionsknoten
			\item Konvertermethode, zum Erstellen eines detaillierten, versendbaren Objektes
		\end{itemize} \\
		Service & \compress \begin{itemize}
			\item Speicherung der Dienst-Eigenschaften
			\item Speicherung der zum Dienst gelisteten Tags
			\item Speicherung je einer Liste der passenden Ein- bzw. Ausgabeformate
		\end{itemize} \\
		\rowcolor[HTML]{E7E7E7}
		Format & \compress \begin{itemize}
			\item Speicherung von Name, Version und Kompatibilitätsgrad
		\end{itemize} \\
		Compatibility & \compress \begin{itemize}
			\item Utilklasse
			\item Methode zum bestimmen der Kompatibilität zweier Dienst
			\item Methode soll für Einzelanfragen und Kanten verwendet werden
		\end{itemize} \\
	\end{tabularx}
	\caption{Klassen des Models}
	\label{table:klassenbeschreibung-a}
\end{table}

\begin{table}
	\begin{tabularx}{\textwidth}{p{.2\textwidth} | X}
		\rowcolor[HTML]{C0C0C0}
		\textbf{Klassenname} & \textbf{Aufgabe} \\
		\rowcolor[HTML]{E7E7E7}
		DetailComp & \compress \begin{itemize}
			\item Speicherung der Node- und Edgeobjekte ohne Verweise auf die Serviceobjekte
			\item Eine Flag, die Auskunft über die Bearbeitungsrechte des anfragenden Users gibt
			\item Listen der User (als SimpleUser) mit Bearbeitungs- bzw Einsichtrecht
			\item Liste der Autoren
		\end{itemize}\\
		\rowcolor[HTML]{E7E7E7}
		Edge & \compress \begin{itemize}
		  \item Besteht aus einem geordneten Paar von Nodeobjekten, dem Eingang- und Ausgangsknoten.
			\item Speicherung einer Liste von Alternativen (Konvertern), die bei Inkompatibilität genutzt werden können.
			\item Speicherung einer Liste der Formate, die die an der Kante beteiligten Dienste kompatibel machen
		\end{itemize}\\
		Node & \compress \begin{itemize}
			\item Speicherung der wichtigsten Daten ohne Referenzen zum Verschicken der Objekte
			\item Speicherung einer künstlichen ID
		\end{itemize}\\
		\rowcolor[HTML]{E7E7E7}
		SimpleComp & \compress \begin{itemize}
			\item Speicherung der elementarsten Daten für die Listenansicht aller Kompositionen
		\end{itemize}\\
		SimpleUser  & \compress \begin{itemize}
			\item Speicherung der elementarsten Daten (ID und Name)
		\end{itemize}\\
		\rowcolor[HTML]{E7E7E7}
		DetailUser & \compress \begin{itemize}
			\item Dient der Bearbeitung der User
			\item Flag um User als Administratoren zu markieren
			\item Flag um Viewer als Administrator zu markieren, um ihm mehr Optionen anzuzueigen
			\item Keine Passwortvermerk
		\end{itemize}\\
		Alternative & \compress \begin{itemize}
			\item Speicherung der möglichen Konverter eventuell Konverterketten (Name, Versionsnummer, ID) um Kompatibilität zu erzeugen
		\end{itemize}\\
		\rowcolor[HTML]{E7E7E7}
		CompatibillityAnswer & \compress \begin{itemize}
			\item Speicherung der Kompatibilität zwischen zwei Diensten
			\item Speicherung der effektiv kompatiblen Formate
			\item Listen von Alternativen
			\item zum antworten auf Einzelanfragen für zwei Dienst
		\end{itemize}\\
	\end{tabularx}
	\caption{Klassen des Models \textit{Fortsetzung}}
\end{table}

\begin{tcolorbox}
Teilt eure Klassendiagramme bitte auf und baut \textbf{kein} einzelnes riesiges Diagramm.
Getter und Setter Methoden müssen hier nicht modelliert werden.
Sie sollten aber der klassischen Namenskonvention folgen, um die Nutzung in Sequenzdiagrammen zu ermöglichen.
\\\\
Auf jedes Diagramm folgt eine Tabelle, in der die Aufgabe \textbf{jeder} Klasse beschrieben wird.
\end{tcolorbox}
