\begin{figure}[h]
	\centering
	\missingfigure{Klassendiagramm}		
	\caption{Klassendiagramm - A}
	\label{fig:klassendiagramm-a}
\end{figure}

\begin{table}[h]
	\centering	
	\begin{tabularx}{\textwidth}{X X}
		\rowcolor[HTML]{C0C0C0} 
		\textbf{Klassenname} & \textbf{Aufgabe} \\
		User & \begin{itemize}
			\item Speicherung der Nutzerdaten, sowie einer künstlichen Datenbank-ID
		\end{itemize}\\
		\rowcolor[HTML]{E7E7E7} 
		Composition & \begin{itemize}
			\item Speicherung der Kompositionen als Graphen, durch Speicherung der Knoten und Kanten
			\item Speicherung des Urhebers und der Nutzenden mit Zugriffs- bzw. Bearbeitungsrechten 
			\item Konvertermethode, zum Erstellen eines reduzierten Objektes
		\end{itemize} \\
		CompositionNode & \begin{itemize}
			\item Speicherung eines Dienstes, für dessen Verwendung im Kompositionsgraphen
			\item Dazu Speicherung einer Position und Aunsdehnung des Knotens
			\item Konvertermethode, zum Erstellen eines reduzierten Objektes
		\end{itemize} \\
		\rowcolor[HTML]{E7E7E7} 
		CompositionEdge & \begin{itemize}
			\item Speicherung einer gerichteten Kompositionskante als Paar von Kompositionsknoten
			\item Bestimmung der Kompatibilität der, an der Kante beteiligten Knoten
			\item Konvertermethode, zum Erstellen eines reduzierten Objektes
		\end{itemize} \\
		Service & \begin{itemize}
			\item Speicherung der Dienst-Eigenschaften 
			\item Speicherung der zum Dienst gelisteten Tags
			\item Speicherung je einer Liste der passenden Ein- bzw- Ausgabeformate 
		\end{itemize} \\
		\rowcolor[HTML]{E7E7E7} 
		Format & \begin{itemize}
			\item Speicherung von Name, Version und Kompatibilitätsgrad
		\end{itemize} \\
	\end{tabularx}
	\caption{Klassen des Models}
	\label{table:klassenbeschreibung-a}
\end{table}

\begin{table}
	\begin{tabularx}{\textwidth}{X X}
		\rowcolor[HTML]{C0C0C0} 
		\textbf{Klassenname} & \textbf{Aufgabe} \\
		SendComp & \begin{itemize}
			\item Speicherung der wichtigsten Daten mit möglichst wenig Referenzen zum Verschicken der Objekte als JSON über REST
		\end{itemize}\\
		\rowcolor[HTML]{E7E7E7}
		ReducedEdge & \begin{itemize}
			\item Speicherung der wichtigsten Daten mit möglichst wenig Referenzen zum Verschicken der Objekte als JSON über REST
		\end{itemize}\\
		ReducedNode & \begin{itemize}
			\item Speicherung der wichtigsten Daten mit möglichst wenig Referenzen zum Verschicken der Objekte als JSON über REST
		\end{itemize}\\
		\rowcolor[HTML]{E7E7E7}
		ListComp & \begin{itemize}
			\item Speicherung der elementarsten Daten für die Listenansicht aller Kompositionen
		\end{itemize}\\
	\end{tabularx}
	\caption{Klassen des Models \textit{Fortsetzung}}
\end{table}

\begin{tcolorbox}
Teilt eure Klassendiagramme bitte auf und baut \textbf{kein} einzelnes riesiges Diagramm.
Getter und Setter Methoden müssen hier nicht modelliert werden.
Sie sollten aber der klassischen Namenskonvention folgen, um die Nutzung in Sequenzdiagrammen zu ermöglichen.
\\\\
Auf jedes Diagramm folgt eine Tabelle, in der die Aufgabe \textbf{jeder} Klasse beschrieben wird.
\end{tcolorbox}