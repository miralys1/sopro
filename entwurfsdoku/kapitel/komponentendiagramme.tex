\begin{figure}[h]
	\centering
	\missingfigure{Komponentendiagramm}
	\caption{Komponentendiagramm - A}
	\label{fig:komponentendiagramm-a}
\end{figure}

\noindent
Die Serverkomponente bzw. das Backend beschreibt die internen Vorgänge der Software.
Diese Komponente ist über drei verschiedene Schnittstellen, welche die unterschiedlichen
Benutzerrechte repräsentieren, erreichbar.\\
In der Komponente ``Benutzerverwaltung'' werden die Daten der Benutzer aus der Datenbank ausgelesen und verarbeitet.
Dadurch kann die Benutzerverwaltung eine Authentifizierungsschnittstelle für die anderen Subkomponenten bereitstellen.
Zudem kann die Rechteverwaltung manipuliert werden, falls über die \textit{Verwalten}-Schnittstelle zugegriffen wird.
Die Komponente ``Diensteverwaltung'' liest die Daten der Dienste aus der Datenbank aus und verarbeitet diese.
Von außen kann auf die Diensteverwaltung durch die Schnittstellen \textit{Bearbeiten} und \textit{Einsehen} zugegriffen werden.
Die Schnittstelle \textit{Bearbeiten} ist nur für Administratoren über die Schnittstelle \textit{Verwalten} zugänglich.
Es werden bei jedem Zugriff die Rechte des zugreifenden Benutzers durch die Schnittstelle \textit{Auth} der Benutzerverwaltung überprüft.\\
Die Komponente ``Kompositionsverwaltung'' stellt die benötigten Funktionen für das Bearbeiten und Einsehen von Kompositionen zur Verfügung und bekommt die Daten dafür von den Komponenten ``Datenbank'' und ``Diensteverwaltung''.
Das Bearbeiten von Kompositionen ist nur über die Schnittstellen \textit{Bearbeiten} und \textit{Verwalten} möglich, das Einsehen auch über das \textit{Einsehen}-Interface.\\
Die Komponente ``Datenbank'' speichert alle nötigen Daten für die Komponenten ``Benutzer-'', ``Dienste-'' und ``Kompositionsverwaltung'' und stellt diese bei Bedarf zur Verfügung.
Zusätzlich kann man über die \textit{Einsehen}-Schnittstelle auch direkt auf die Authentifizierung zugreifen um eine initiale Registrierung und den Login zu ermöglichen.\\
\\
Die Komponente ``Web'' mit der Subkomponente ``View Verwaltung'' ist für die Darstellung in der Webapplikation zuständig.
Da man über die Weboberfläche verwalten, bearbeiten und einsehen können soll, ist die Web-Komponente auch mit all diesen Schnittstellen verbunden.\\
Die Komponente ``App'' mit der Subkomponente ``View'' dient zur Darstellung der angefragten Daten in der App.
In der App können nur Kompositionen eingesehen werden, daher ist diese nur mit der \textit{Einsehen}-Schnittstelle verbunden.
