\section{Dokumentaufbau}\label{sec:dokumentaufbau}
\subsection{Ziel des Dokuments}
Ziel des Dokuments ist die Dokumentation der Entwicklung des SWARM Composers. 
Die hier aufgeführten Diagramme und Erklärungen sollen neuen Teammitgliedern den Einstieg in das Projekt und zukünftigen Entwicklern die Einarbeitung in die Software erleichtern.
Die frühzeitige Planung der Struktur der Software führt zu einem reibungsloseren Entwicklungsprozess. 
Außerdem finden sich hier konkrete Absprachen der kleineren Teilprojekte, um die Entwicklung entkoppelter Komponenten zu ermöglichen.
\subsection{Aufbau}
In den Kapiteln 2 und 3 wird je ein plattformübergreifendes Komponenten- bzw. Verteilungsdiagramm dargestellt. 
Darauf folgt in Kapitel 4 eine Sammlung von Klassendiagrammen, die die Code-Struktur auf den jeweiligen Plattformen erklären.
In Kapitel 5 werden die wichtigsten Methodenaufrufe und Abläufe, die in den Klassendiagrammen zu finden sind, als Sequenzdiagramme modelliert.


\section{Zweckbestimmung}\label{sec:zweckbestimmung}
Der SWARM Composer dient dazu, unterschiedliche Software für Bauprojekte in Bezug auf ihre Kompatibilität bezüglich der Ein- und Ausgabeformate zu überprüfen. 
Die Software ist Teil des Forschungsprojektes SWARM des Unternehmens adesso. 
Der SWARM Composer besteht aus zwei Teilen: einem Webserver und einer App. 
Auf dem Webserver können Benutzer und Benutzerinnen Dienste zu Kompositionen zusammenfassen und auf Kompatibilität überprüfen.
In der App können Kompositionen grafisch präsentiert und per PDF verschickt werden.
\section{Entwicklungsumgebung}\label{sec:entwicklungsumgebung}
In der untenstehenden Tabelle finden sich grundlegende Frameworks, Bibliotheken, Tools und Sprachen, die für die Entwicklung im Rahmen dieses Projektes erforderlich sind.


\begin{table}[h]
	\centering
	\begin{tabularx}{\textwidth}{l l X}
		\rowcolor[HTML]{C0C0C0}
		\textbf{Software} & \textbf{Version} & \textbf{URL} \\
		\rowcolor[HTML]{E7E7E7}
		Maven & 4.0.0 & \url{maven.apache.org/POM/4.0.0} \\
		Spring & 2.0.4 & \url{spring.io} \\
		\rowcolor[HTML]{E7E7E7}
		Android Studio & 3.1.4 & \url{https://developer.android.com/studio/}\\
		Tomcat & 8.0.53 & \url{http://tomcat.apache.org} \\
		\rowcolor[HTML]{E7E7E7}
		git & 2.17.1 & \url{https://git-scm.com/downloads} \\
		node.js & 8.11.4 & \url{https://nodejs.org/en/download/} \\
		\rowcolor[HTML]{E7E7E7}
		Visual Paradigm & 15.1 & \url{https://www.visual-paradigm.com/download/}\\
		Thymeleaf & 3.9.9 & \url{http://www.thymeleaf.org/download.html} \\
		\rowcolor[HTML]{E7E7E7}
		vue.js & 2.5.2 & \url{https://vuejs.org} \\
		Bootstrap & 4.1.1 & \url{https://getbootstrap.com} \\
		\rowcolor[HTML]{E7E7E7}
		Javascript & ES6 & - \\
		Java Development Kit & 8u144 & \url{http://www.oracle.com/technetwork/java/javase/downloads/index.html}

	\end{tabularx}
	\caption{Enwicklungsumgebung}
	\label{table:entwicklungsumgebung}
\end{table}