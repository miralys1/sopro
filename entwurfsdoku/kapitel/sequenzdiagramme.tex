	Das dynamische Verhalten des Systems wird mittels Sequenzdiagrammen modelliert.
	Hier wird zunächst ein grobes geräteübergreifendes Diagramm vorgestellt.
	Die restlichen Sequenzdiagramme beziehen sich nur auf wichtige Methoden in den gekapselten Ökosystemen APP, Website und Backend. Deswegen werden Lost und Found Messages verwendet, um die Kommunikation mit dem Server darzustellen. 
	
\section*{Geräteübergreifendes Sequenzdiagramm}

\begin{figure}[h]
	\centering
	%\includegraphics[width= \linewidth ]{img/SDDetail.svg}
	\missingfigure{Sequenzdiagramm}		
	\caption{Sequenzdiagramm}
	\label{fig:sequenz-a}
\end{figure}
\noindent
Der Benutzer öffnet die App, wodurch die Methode onCreate() der ListActivity aufgerufen wird. Zum Start wird ein HTTPS-Request an den Server gesendet. Der Server fragt die Daten aus der Datenbank ab und erstellt dazu ein versendbares Objekt in Form einer Liste von SimpleComp. Dies wird im Rahmen der Beantwortung des GET-Requests an die App geschickt. Die App verarbeitet die erhaltenen Daten, sodass sie dem Benutzer angezeigt werden können.

\section*{Sequenzdiagramme der App}
\subsection*{Sequenzdiagramm: Öffnen der ListActivity}

\begin{figure}[h]
	\centering
%\includegraphics[width= \linewidth ]{img/SDListAct.svg}
	\missingfigure{Sequenzdiagramm}			
	\caption{Sequenzdiagramm - Öffnen der ListActivity}
	\label{fig:sequenz-a}
\end{figure}
\noindent
Dieses Sequenzdiagramm zeigt den Vorgang, der abläuft, wenn die ListActivity initial gestartet wird und mit einer Liste aus Kompositionseinträgen zu füllen ist.
Da der Nutzer noch nicht eingeloggt ist, wird eine requestList()-Anfrage ohne Token gestartet. Diese wird in einem eigenen Thread ausgeführt, damit der UI-Thread nicht blockiert. Die Methode requestList() führt einen HTTPS-Request an das Backend aus. In der Zeit, in der die ListActivity die Antwort noch nicht erhalten hat, zeigt sie mithilfe von showLoading() eine Ladeanimation an.

\subsection*{Sequenzdiagramm: Klicken auf Kompositionseintrag}

\begin{figure}[h]
	\centering
	%\includegraphics[width= \linewidth ]{img/SDDetail.svg}
	\missingfigure{Sequenzdiagramm}		
	\caption{Sequenzdiagramm - Klick auf Kompositionseintrag}
	\label{fig:sequenz-a}
\end{figure}
\noindent
Dieses Sequenzdiagramm beginnt damit, dass der Nutzer auf ein Listenelement tippt, um die Detailansicht aufzurufen. 
Dadruch wird die Methode startDetailActivity() aufgerufen, die die Klasse ServerCommunication veranlasst, im LokalCache nach der gewünschten Komposition zu suchen.
Falls diese noch nicht im Cache ist, wird ein HTTPS-Request an den Server gesendet.
In der Antwort sind dann alle nötigen Details der Komposition enthalten.
Diese werden dann im LokalCache abgelegt und an die ListActivity zurückgegeben.
Damit wird dann ein Intent erstellt und die DetailActivity gestartet.
\\
Da LocalCache ein Singleton ist, muss zu Beginn die Instanz abgerufen werden. Wir planen die Instanz vorzudefinieren (eager instantiation), wodurch der Fall, dass die Instanz erst erzeugt werden muss, nie eintritt.
\\
Die Methode httpsRequest() wird nicht implementiert, sondern dient hier zur Abstraktion. In der Implementation wird diese Abfrage asynchron ablaufen und in einem anderen Thread laufen, so dass der UI-Thread nicht blockiert wird. 